\documentclass{article}\usepackage[]{graphicx}\usepackage[]{xcolor}
% maxwidth is the original width if it is less than linewidth
% otherwise use linewidth (to make sure the graphics do not exceed the margin)
\makeatletter
\def\maxwidth{ %
  \ifdim\Gin@nat@width>\linewidth
    \linewidth
  \else
    \Gin@nat@width
  \fi
}
\makeatother

\definecolor{fgcolor}{rgb}{0.345, 0.345, 0.345}
\newcommand{\hlnum}[1]{\textcolor[rgb]{0.686,0.059,0.569}{#1}}%
\newcommand{\hlsng}[1]{\textcolor[rgb]{0.192,0.494,0.8}{#1}}%
\newcommand{\hlcom}[1]{\textcolor[rgb]{0.678,0.584,0.686}{\textit{#1}}}%
\newcommand{\hlopt}[1]{\textcolor[rgb]{0,0,0}{#1}}%
\newcommand{\hldef}[1]{\textcolor[rgb]{0.345,0.345,0.345}{#1}}%
\newcommand{\hlkwa}[1]{\textcolor[rgb]{0.161,0.373,0.58}{\textbf{#1}}}%
\newcommand{\hlkwb}[1]{\textcolor[rgb]{0.69,0.353,0.396}{#1}}%
\newcommand{\hlkwc}[1]{\textcolor[rgb]{0.333,0.667,0.333}{#1}}%
\newcommand{\hlkwd}[1]{\textcolor[rgb]{0.737,0.353,0.396}{\textbf{#1}}}%
\let\hlipl\hlkwb

\usepackage{framed}
\makeatletter
\newenvironment{kframe}{%
 \def\at@end@of@kframe{}%
 \ifinner\ifhmode%
  \def\at@end@of@kframe{\end{minipage}}%
  \begin{minipage}{\columnwidth}%
 \fi\fi%
 \def\FrameCommand##1{\hskip\@totalleftmargin \hskip-\fboxsep
 \colorbox{shadecolor}{##1}\hskip-\fboxsep
     % There is no \\@totalrightmargin, so:
     \hskip-\linewidth \hskip-\@totalleftmargin \hskip\columnwidth}%
 \MakeFramed {\advance\hsize-\width
   \@totalleftmargin\z@ \linewidth\hsize
   \@setminipage}}%
 {\par\unskip\endMakeFramed%
 \at@end@of@kframe}
\makeatother

\definecolor{shadecolor}{rgb}{.97, .97, .97}
\definecolor{messagecolor}{rgb}{0, 0, 0}
\definecolor{warningcolor}{rgb}{1, 0, 1}
\definecolor{errorcolor}{rgb}{1, 0, 0}
\newenvironment{knitrout}{}{} % an empty environment to be redefined in TeX

\usepackage{alltt}
\usepackage[margin=1.0in]{geometry} % To set margins
\usepackage{amsmath}  % This allows me to use the align functionality.
                      % If you find yourself trying to replicate
                      % something you found online, ensure you're
                      % loading the necessary packages!
\usepackage{amsfonts} % Math font
\usepackage{fancyvrb}
\usepackage{hyperref} % For including hyperlinks
\usepackage[shortlabels]{enumitem}% For enumerated lists with labels specified
                                  % We had to run tlmgr_install("enumitem") in R
\usepackage{float}    % For telling R where to put a table/figure
\usepackage{natbib}        %For the bibliography
\bibliographystyle{apalike}%For the bibliography
\IfFileExists{upquote.sty}{\usepackage{upquote}}{}
\begin{document}
  \begin{enumerate}
  %%%%%%%%%%%%%%%%%%%%%%%%%%%%%%%%%%%%%%%%%%%%%%%%%%%%%%%%%%%%%%%%%%%%%%%%%%%%%%
  % Question 1
  %%%%%%%%%%%%%%%%%%%%%%%%%%%%%%%%%%%%%%%%%%%%%%%%%%%%%%%%%%%%%%%%%%%%%%%%%%%%%%
    \item Write a \texttt{pois.prob()} function that computes $P(X=x)$, 
    $P(X \neq x)$, $P(X<x)$, $P(X \leq x)$, $P(X > x)$, and $P(X \geq x).$ Enable the user to specify the rate parameter $\lambda$.
\begin{knitrout}\scriptsize
\definecolor{shadecolor}{rgb}{0.969, 0.969, 0.969}\color{fgcolor}\begin{kframe}
\begin{alltt}
\hldef{pois.prob} \hlkwb{<-} \hlkwa{function}\hldef{(}\hlkwc{x}\hldef{,}             \hlcom{#x = number of event and occurrences}
                      \hlkwc{lambda}\hldef{,}        \hlcom{#lambda = the rate parameter}
                      \hlkwc{type} \hldef{=} \hlsng{"<="}\hldef{) \{} \hlcom{#type = determines the type of probability calculated}
  \hlcom{#function purpose:}
  \hlcom{#uses dpois and ppois to conditionally return the correct probability}

  \hlkwa{if}\hldef{(type} \hlopt{==} \hlsng{"="}\hldef{)\{}
    \hlcom{#P(X=x), calculates the probability of exactly observing x occurrences within a poisson distribution}
    \hldef{prob} \hlkwb{=} \hlkwd{dpois}\hldef{(}\hlkwc{x} \hldef{= x,} \hlkwc{lambda} \hldef{= lambda)}
  \hldef{\}} \hlkwa{else if}\hldef{(type} \hlopt{==} \hlsng{"!="}\hldef{)\{}
    \hlcom{#P(X!=x), calculates the probability of observing all occurrences except x within a poisson distribution}
    \hldef{prob} \hlkwb{=} \hlnum{1} \hlopt{-} \hlkwd{dpois}\hldef{(}\hlkwc{x} \hldef{= x,} \hlkwc{lambda} \hldef{= lambda)}
  \hldef{\}} \hlkwa{else if}\hldef{(type} \hlopt{==} \hlsng{">"}\hldef{) \{}
    \hlcom{#P(X>x), calculates the probability of observing more than x occurrences within a poisson distribution}
    \hldef{prob} \hlkwb{=} \hlnum{1} \hlopt{-} \hlkwd{ppois}\hldef{(}\hlkwc{q} \hldef{= x,} \hlkwc{lambda} \hldef{= lambda)}
  \hldef{\}} \hlkwa{else if}\hldef{(type} \hlopt{==} \hlsng{"<"}\hldef{) \{}
    \hlcom{#P(X<x), calculates the probability of observing less than x occurrences within a poisson distribution}
    \hldef{prob} \hlkwb{=} \hlkwd{ppois}\hldef{(}\hlkwc{q} \hldef{= x} \hlopt{-} \hlnum{1}\hldef{,} \hlkwc{lambda} \hldef{= lambda)}
  \hldef{\}} \hlkwa{else if}\hldef{(type} \hlopt{==} \hlsng{">="}\hldef{) \{}
    \hlcom{#P(X>=x), calculates the probability of observing at least x occurrences within a poisson distribution}
    \hldef{prob} \hlkwb{=} \hlnum{1} \hlopt{-} \hlkwd{ppois}\hldef{(}\hlkwc{q} \hldef{= x} \hlopt{-} \hlnum{1}\hldef{,} \hlkwc{lambda} \hldef{= lambda)}
  \hldef{\}} \hlkwa{else if}\hldef{(type} \hlopt{==} \hlsng{"<="}\hldef{) \{}
    \hlcom{#P(X<=x), calculates the probability of observing at most x occurrences within a poisson distribution}
    \hldef{prob} \hlkwb{=} \hlkwd{ppois}\hldef{(}\hlkwc{q} \hldef{= x,} \hlkwc{lambda} \hldef{= lambda)}
  \hldef{\}} \hlkwa{else} \hldef{\{}
    \hlcom{#If parameters entered are incorrect, stop the function}
    \hlkwd{stop}\hldef{(}\hlsng{"Enter valid parameters (numericals for x and lambda, inequality for type)."}\hldef{)}
  \hldef{\}}

  \hlkwd{return}\hldef{(prob)} \hlcom{#return the probability}
\hldef{\}}
\end{alltt}
\end{kframe}
\end{knitrout}
  %%%%%%%%%%%%%%%%%%%%%%%%%%%%%%%%%%%%%%%%%%%%%%%%%%%%%%%%%%%%%%%%%%%%%%%%%%%%%%
  % Question 2
  %%%%%%%%%%%%%%%%%%%%%%%%%%%%%%%%%%%%%%%%%%%%%%%%%%%%%%%%%%%%%%%%%%%%%%%%%%%%%%
    \item Write a \texttt{beta.prob()} function that computes $P(X=x)$, 
    $P(X \neq x)$, $P(X<x)$, $P(X \leq x)$, $P(X > x)$, and $P(X \geq x)$
    for a beta distribution. Enable the user to specify the shape parameters
    $\alpha$ and $\beta$.
\begin{knitrout}\scriptsize
\definecolor{shadecolor}{rgb}{0.969, 0.969, 0.969}\color{fgcolor}\begin{kframe}
\begin{alltt}
\hldef{beta.prob} \hlkwb{<-} \hlkwa{function}\hldef{(}\hlkwc{x}\hldef{,}           \hlcom{#x = number of event/occurences}
                      \hlkwc{alpha}\hldef{,}       \hlcom{#alpha = the alpha parameter}
                      \hlkwc{beta}\hldef{,}        \hlcom{#beta = the beta parameter}
                      \hlkwc{type} \hldef{=} \hlsng{"<="}\hldef{) \{} \hlcom{#type = the type of probability to be calculated}
  \hlcom{#function purpose:}
  \hlcom{#Use dbeta and pbeta to conditionally return the correct probability}

  \hlkwa{if}\hldef{(type} \hlopt{==} \hlsng{"="}\hldef{)\{}
    \hlcom{#P(X=x), calculates the probability of exactly observing x occurrences within a beta distribution}
    \hlcom{#(This js always 0)}
    \hldef{prob} \hlkwb{=} \hlnum{0}
  \hldef{\}} \hlkwa{else if}\hldef{(type} \hlopt{==} \hlsng{"!="}\hldef{)\{}
    \hlcom{#P(X!=x), calculates the probability of observing all occurrences except x within a beta distribution}
    \hlcom{#(This is always 1)}
    \hldef{prob} \hlkwb{=} \hlnum{1}
  \hldef{\}} \hlkwa{else if}\hldef{(type} \hlopt{==} \hlsng{">"}\hldef{) \{}
    \hlcom{#P(X>x), calculates the probability of observing more than x occurrences within a beta distribution}
    \hldef{prob} \hlkwb{=} \hlnum{1} \hlopt{-} \hlkwd{pbeta}\hldef{(}\hlkwc{q} \hldef{= x,} \hlkwc{shape1} \hldef{= alpha,} \hlkwc{shape2} \hldef{= beta)}
  \hldef{\}} \hlkwa{else if}\hldef{(type} \hlopt{==} \hlsng{"<"}\hldef{) \{}
    \hlcom{#P(X<x), calculates the probability of observing less than x occurrences within a beta distribution}
    \hldef{prob} \hlkwb{=} \hlkwd{pbeta}\hldef{(}\hlkwc{q} \hldef{= x,} \hlkwc{shape1} \hldef{= alpha,} \hlkwc{shape2} \hldef{= beta)}
  \hldef{\}} \hlkwa{else if}\hldef{(type} \hlopt{==} \hlsng{">="}\hldef{) \{}
    \hlcom{#P(X>=x), calculates the probability of observing at least x occurrences within a beta distribution}
    \hldef{prob} \hlkwb{=} \hlnum{1} \hlopt{-} \hlkwd{pbeta}\hldef{(}\hlkwc{q} \hldef{= x,} \hlkwc{shape1} \hldef{= alpha,} \hlkwc{shape2} \hldef{= beta)}
  \hldef{\}} \hlkwa{else if}\hldef{(type} \hlopt{==} \hlsng{"<="}\hldef{) \{}
    \hlcom{#P(X<=x), calculates the probability of observing at most x occurrences within a beta distribution}
    \hldef{prob} \hlkwb{=} \hlkwd{pbeta}\hldef{(}\hlkwc{q} \hldef{= x,} \hlkwc{shape1} \hldef{= alpha,} \hlkwc{shape2} \hldef{= beta)}
  \hldef{\}} \hlkwa{else} \hldef{\{}
    \hlcom{#If parameters entered are incorrect, stop the function}
    \hlkwd{stop}\hldef{(}\hlsng{"Enter valid parameters (numericals for x; alpha; and beta, inequality for type)."}\hldef{)}
  \hldef{\}}

  \hlkwd{return}\hldef{(prob)} \hlcom{#return the probability}
\hldef{\}}
\end{alltt}
\end{kframe}
\end{knitrout}
\end{enumerate}
\bibliography{bibliography}
\end{document}
